\documentclass[a4paper,8pt]{extarticle}
\usepackage[left=0.5cm, right=0.5cm, top=0.5cm, bottom=0.5cm, twoside, headsep=0.15in, includehead, includefoot]{geometry}
\usepackage[utf8]{inputenc}
\usepackage[T1]{fontenc}
\usepackage{Alegreya}
\usepackage[tracking]{microtype}
\usepackage[compact]{titlesec}
\usepackage{xskak}
\xskakset{style=UF}
\usepackage{graphicx}
\usepackage{float}
\usepackage{framed}
\usepackage{xcolor}

\usepackage{hyperref}
\usepackage{fancyhdr}
\pagestyle{fancy}
\fancyhf{}

\definecolor{beansbg}{HTML}{FFFCF0}
\definecolor{beanstext}{HTML}{100F0F}
\definecolor{beansaccent}{HTML}{6F6E69}
\definecolor{beanslight}{HTML}{E6E4D9}

\pagecolor{beansbg}
\color{beanstext}

\renewcommand{\thesection}{\arabic{section}.}

\hypersetup{
    colorlinks=true,
    linkcolor=beansaccent,
    urlcolor=beansaccent,
    citecolor=beansaccent
}

\rhead{\color{beanstext}Page \thepage}
\lhead{\color{beanstext}Common chess openings (revision 1.2.2, last edited: 2025-09-07) - chessbeans edition}
\rfoot{\color{beanstext}Source: \url{https://github.com/papabeans1/theory-of-chess}}

\newcommand{\openingname}[1]{\emph{\color{beanstext}#1}}
\styleA
\begin{document}
\author{papabeans1}
\title{Chess openings - chessbeans edition}
\section{Open games}
\input{src/open}
\section{Closed games}
\input{src/closed}
\section{Semi-open games}
\input{src/semi}
\section{Indian defences}
\input{src/indian}
\section{Flank openings}
\begin{minipage}[t]{.175\linewidth}
\fenboard{rnbqkbnr/pppppppp/8/8/2P5/8/PP1PPPPP/RNBQKBNR b KQkq c3 0 1}
\raggedright
\begin{center}
\scalebox{.560}{\showboard}
\end{center}
\newgame
%FEN@rnbqkbnr/pppppppp/8/8/2P5/8/PP1PPPPP/RNBQKBNR b KQkq c3 0 1
\openingname{English Opening}: \mainline{1. c4} (diagram).
Then \mainline{1...Nf6} \openingname{Anglo-Indian Defence}, \variation{1...e5} \openingname{Reversed Sicilian}, \variation{1...e6} \openingname{Agincourt Defence} or \variation{1...c5} \openingname{Symmetrical Variation}.

\vspace{2mm}
\end{minipage}
\hspace{5mm}
\begin{minipage}[t]{.175\linewidth}
\fenboard{rnbqk1nr/ppp2ppp/3b4/8/8/8/PPPPP1PP/RNBQKBNR w KQkq - 0 4}
\raggedright
\begin{center}
\scalebox{.560}{\showboard}
\end{center}
\newgame
%FEN@rnbqk1nr/ppp2ppp/3b4/8/8/8/PPPPP1PP/RNBQKBNR w KQkq - 0 4
\openingname{Bird's Opening}: \mainline{1. f4}.

\openingname{Leningrad Variation}: \mainline{1...d5 2. Nf3 g6 3. g3 Bg7 4. Bg2}.
\openingname{From's Gambit}: \variation{1...e5 2. fxe5 d6 3. exd6 Bxd6} (diagram).
\vspace{2mm}
\end{minipage}
\hspace{5mm}


\end{document}
