\documentclass[a4paper,8pt]{extarticle}
\usepackage[left=0.5cm, right=0.5cm, top=0.5cm, bottom=0.5cm, twoside, headsep=0.15in, includehead, includefoot]{geometry}
\usepackage[utf8]{inputenc}
\usepackage[T1]{fontenc}
\usepackage{Alegreya}
\usepackage[tracking]{microtype}
\usepackage[compact]{titlesec}
\usepackage{xskak}
\xskakset{style=UF}
\usepackage{graphicx}
\usepackage{float}
\usepackage{framed}
\usepackage{xcolor}

\usepackage{hyperref}
\usepackage{fancyhdr}
\pagestyle{fancy}
\fancyhf{}

\definecolor{beansbg}{HTML}{FFFCF0}
\definecolor{beanstext}{HTML}{100F0F}
\definecolor{beansaccent}{HTML}{6F6E69}
\definecolor{beanslight}{HTML}{E6E4D9}

\pagecolor{beansbg}
\color{beanstext}

\renewcommand{\thesection}{\arabic{section}.}

\hypersetup{
    colorlinks=true,
    linkcolor=beansaccent,
    urlcolor=beansaccent,
    citecolor=beansaccent
}

\rhead{\color{beanstext}Page \thepage}
\lhead{\color{beanstext}Common chess openings (revision 1.2.2, last edited: 2025-09-07) - chessbeans edition}
\rfoot{\color{beanstext}Source: \url{https://github.com/papabeans1/theory-of-chess}}

\newcommand{\openingname}[1]{\emph{\color{beanstext}#1}}
\styleA
\begin{document}
\author{papabeans1}
\title{Chess openings - chessbeans edition}
\section{Open games}
\begin{minipage}[t]{.175\linewidth}
\fenboard{r1bq1rk1/2p1bppp/p1np1n2/1p2p3/4P3/1BP2N2/PP1P1PPP/RNBQR1K1 w - - 1 9}
\raggedright
\begin{center}
\scalebox{.560}{\showboard}
\end{center}
\newgame
%FEN@r1bq1rk1/2p1bppp/p1np1n2/1p2p3/4P3/1BP2N2/PP1P1PPP/RNBQR1K1 w - - 1 9
\openingname{Ruy Lopez}: \mainline{1. e4 e5 2. Nf3 Nc6 3. Bb5}.
\openingname{Morphy Defence}: \mainline{3...a6}.
Main line: \mainline{4. Ba4 Nf6 5. O-O Be7 6. Re1 b5 7. Bb3 d6 8. c3 O-O} (diagram).
\vspace{2mm}
\end{minipage}
\hspace{5mm}
\begin{minipage}[t]{.175\linewidth}
\fenboard{r1bk1b1r/ppp2ppp/2p5/4Pn2/8/5N2/PPP2PPP/RNB2RK1 w - - 0 9}
\raggedright
\begin{center}
\scalebox{.560}{\showboard}
\end{center}
\newgame
%FEN@r1bk1b1r/ppp2ppp/2p5/4Pn2/8/5N2/PPP2PPP/RNB2RK1 w - - 0 9
\openingname{Ruy Lopez}: \mainline{1. e4 e5 2. Nf3 Nc6 3. Bb5}.
\openingname{Berlin Defence}: \mainline{3...Nf6}.
Main line: \mainline{4. O-O Nxe4 5. d4 Nd6 6. Bxc6 dxc6 7. dxe5 Nf5 8. Qxd8+ Kxd8} (diagram).
\vspace{2mm}
\end{minipage}
\hspace{5mm}
\begin{minipage}[t]{.175\linewidth}
\fenboard{r1bqk2r/pppp1ppp/2n2n2/2b1p3/2B1P3/2PP1N2/PP3PPP/RNBQK2R b KQkq - 0 5}
\raggedright
\begin{center}
\scalebox{.560}{\showboard}
\end{center}
\newgame
%FEN@r1bqk2r/pppp1ppp/2n2n2/2b1p3/2B1P3/2PP1N2/PP3PPP/RNBQK2R b KQkq - 0 5
\openingname{Italian Game}: \mainline{1. e4 e5 2. Nf3 Nc6 3. Bc4}.
\openingname{Giuoco Piano}: \mainline{3...Bc5}.
\openingname{Giuoco Pianissimo}: \mainline{4. c3 Nf6 5. d3} (diagram) or \openingname{Evans Gambit}: \variation{4. b4} followed by \variation{4...Bxb4 5. c3}.
\vspace{2mm}
\end{minipage}
\hspace{5mm}
\begin{minipage}[t]{.175\linewidth}
\fenboard{r1bqkb1r/ppp2ppp/5n2/n2Pp1N1/2B5/8/PPPP1PPP/RNBQK2R w KQkq - 1 6}
\raggedright
\begin{center}
\scalebox{.560}{\showboard}
\end{center}
\newgame
%FEN@r1bqkb1r/ppp2ppp/5n2/n2Pp1N1/2B5/8/PPPP1PPP/RNBQK2R w KQkq - 1 6
\openingname{Italian game}: \mainline{1. e4 e5 2. Nf3 Nc6 3. Bc4}.
\openingname{Two Knights Defence}: \mainline{3...Nf6}.
Main line: \mainline{4. Ng5 d5 5. exd5 Na5} (diagram).
\openingname{Modern Bishop’s Opening}: \variation{4. d3} followed by \variation{4...Be7 5. O-O O-O 6. Re1 d6}.
\vspace{2mm}
\end{minipage}
\hspace{5mm}
\begin{minipage}[t]{.175\linewidth}
\fenboard{r1bqkbnr/pppp1ppp/2n5/8/3NP3/8/PPP2PPP/RNBQKB1R b KQkq - 0 4}
\raggedright
\begin{center}
\scalebox{.560}{\showboard}
\end{center}
\newgame
%FEN@r1bqkbnr/pppp1ppp/2n5/8/3NP3/8/PPP2PPP/RNBQKB1R b KQkq - 0 4
\openingname{Scotch game}: \mainline{1. e4 e5 2. Nf3 Nc6 3. d4 exd4}.
Main line: \mainline{4. Nxd4} (diagram), followed by \openingname{Classical}: \variation{4...Bc5} or \openingname{Schmidt Variation}: \variation{4...Nf6}.
Other lines: \openingname{Scotch Gambit} \variation{4. Bc4}, \openingname{Göring Gambit} \variation{4. c3}.
\vspace{2mm}
\end{minipage}
\newline
\begin{minipage}[t]{.175\linewidth}
\fenboard{r1bqk2r/pppp1ppp/2n2n2/1B2p3/1b2P3/2N2N2/PPPP1PPP/R1BQK2R w KQkq - 6 5}
\raggedright
\begin{center}
\scalebox{.560}{\showboard}
\end{center}
\newgame
%FEN@r1bqk2r/pppp1ppp/2n2n2/1B2p3/1b2P3/2N2N2/PPPP1PPP/R1BQK2R w KQkq - 6 5
\openingname{Four Knights Game}: \mainline{1. e4 e5 2. Nf3 Nc6 3. Nc3 Nf6}.
Spanish lines: \openingname{Double Spanish} \mainline{4. Bb5 Bb4} (diagram), \openingname{Rubinstein} \variation{4...Nd4} or \openingname{Classical Variation} \variation{4...Bc5}.
\vspace{2mm}
\end{minipage}
\hspace{5mm}
\begin{minipage}[t]{.175\linewidth}
\fenboard{r1bqk2r/p1pp1ppp/2p2n2/8/1b2P3/2NB4/PPP2PPP/R1BQK2R b KQkq - 1 7}
\raggedright
\begin{center}
\scalebox{.560}{\showboard}
\end{center}
\newgame
%FEN@r1bqk2r/p1pp1ppp/2p2n2/8/1b2P3/2NB4/PPP2PPP/R1BQK2R b KQkq - 1 7
\openingname{Four Knights Game}: \mainline{1. e4 e5 2. Nf3 Nc6 3. Nc3 Nf6}.
\openingname{Scotch Four Knights Game}: \mainline{4. d4}, main line: \mainline{4...exd4 5. Nxd4 Bb4 6. Nxc6 bxc6 7. Bd3} (diagram).
\vspace{2mm}
\end{minipage}
\hspace{5mm}
\begin{minipage}[t]{.175\linewidth}
\fenboard{rnbqkb1r/ppp2ppp/3p4/8/4n3/5N2/PPPP1PPP/RNBQKB1R w KQkq - 0 5}
\raggedright
\begin{center}
\scalebox{.560}{\showboard}
\end{center}
\newgame
%FEN@rnbqkb1r/ppp2ppp/3p4/8/4n3/5N2/PPPP1PPP/RNBQKB1R w KQkq - 0 5
\openingname{Petrov's Defence}: \mainline{1. e4 e5 2. Nf3 Nf6}.
\openingname{Classical Variation}: \mainline{3. Nxe5} followed by \mainline{3...d6 4. Nf3 Nxe4} (diagram).
\openingname{Steinitz Attack}: \variation{3. d4} followed by \variation{3...Nxe4} or \variation{3...exd4 4. e5 Ne4}.
\vspace{2mm}
\end{minipage}
\hspace{5mm}
\begin{minipage}[t]{.175\linewidth}
\fenboard{rnbqkbnr/ppp3pp/3p4/4pp2/3PP3/5N2/PPP2PPP/RNBQKB1R w KQkq f6 0 4}
\raggedright
\begin{center}
\scalebox{.560}{\showboard}
\end{center}
\newgame
%FEN@rnbqkbnr/ppp3pp/3p4/4pp2/3PP3/5N2/PPP2PPP/RNBQKB1R w KQkq f6 0 4
\openingname{Philidor Defence}: \mainline{1. e4 e5 2. Nf3 d6}.
Main line: \mainline{3. d4} followed by \openingname{Philidor Countergambit} \mainline{3...f5} (diagram).
Other lines: \openingname{Exchange} \variation {3...exd4}, \openingname{Nimzowitch} \variation {3...Nf6} or \openingname{Hanham Variation} \variation {3...Nd7}.
\vspace{2mm}
\end{minipage}
\hspace{5mm}
\begin{minipage}[t]{.175\linewidth}
\fenboard{rnbqkbnr/pppp1p1p/8/6p1/4Pp1P/5N2/PPPP2P1/RNBQKB1R b KQkq h3 0 4}
\raggedright
\begin{center}
\scalebox{.560}{\showboard}
\end{center}
\newgame
%FEN@rnbqkbnr/pppp1p1p/8/6p1/4Pp1P/5N2/PPPP2P1/RNBQKB1R b KQkq h3 0 4
\openingname{King's Gambit Accepted}: \mainline{1. e4 e5 2. f4 exf4}.
\openingname{Paris Attack}: \mainline{3. Nf3 g5 4. h4} (diagram).
Other lines: \variation{4. Bc4}, \variation{4. Nc3}.
\openingname{Modern Defence}: \variation{3...d5}.
\openingname{Bishop's Gambit}: \variation{3. Bc4} followed by \variation{3...Nf6} or \variation{3...d5}.
\vspace{2mm}
\end{minipage}
\newline
\begin{minipage}[t]{.175\linewidth}
\fenboard{rnbqkbnr/ppp2ppp/8/3Pp3/5P2/8/PPPP2PP/RNBQKBNR b KQkq - 0 3}
\raggedright
\begin{center}
\scalebox{.560}{\showboard}
\end{center}
\newgame
%FEN@rnbqkbnr/ppp2ppp/8/3Pp3/5P2/8/PPPP2PP/RNBQKBNR b KQkq - 0 3
\openingname{King's Gambit Declined}: \mainline{1. e4 e5 2. f4}.
\openingname{Falkbeer Countergambit}: \mainline{2...d5 3. exd5} (diagram) followed by \mainline{3...e4} or \variation{3...c6}.
\openingname{Classical Variation}: \variation{2...Bc5 3. Nf3 d6}.
\vspace{2mm}
\end{minipage}
\hspace{5mm}
\begin{minipage}[t]{.175\linewidth}
\fenboard{r1bqkbnr/pppp1ppp/2n5/8/4P3/4Q3/PPP2PPP/RNB1KBNR b KQkq - 2 4}
\raggedright
\begin{center}
\scalebox{.560}{\showboard}
\end{center}
\newgame
%FEN@r1bqkbnr/pppp1ppp/2n5/8/4P3/4Q3/PPP2PPP/RNB1KBNR b KQkq - 2 4
\openingname{Center Game}: \mainline{1. e4 e5 2. d4 exd4}.
Universal sequence is \mainline{3. Qxd4 Nc6} and \openingname{Paulsen's attack}: \mainline{4. Qe3} (diagram).
\vspace{2mm}
\end{minipage}
\hspace{5mm}
\begin{minipage}[t]{.175\linewidth}
\fenboard{rnbqkbnr/pppp1ppp/8/8/4P3/2N5/PP3PPP/R1BQKBNR b KQkq - 0 4}
\raggedright
\begin{center}
\scalebox{.560}{\showboard}
\end{center}
\newgame
%FEN@rnbqkbnr/pppp1ppp/8/8/4P3/2N5/PP3PPP/R1BQKBNR b KQkq - 0 4
\openingname{Danish gambit accepted}: \mainline{1. e4 e5 2. d4 exd4 3. c3 dxc3}.
\openingname{Alekhine variation}: \mainline{4. Nxc3} (diagram).
\openingname{Lindehn's continuation} \variation{4. Bc4}.
\variation{3...d6}, \variation{3...Qe7} or \variation{3...d5} to decline.
\vspace{2mm}
\end{minipage}
\hspace{5mm}


\section{Closed games}
\begin{minipage}[t]{.175\linewidth}
\fenboard{rnbqkb1r/1p3ppp/p3pn2/2p5/2BP4/4PN2/PP3PPP/RNBQ1RK1 w kq - 0 7}
\raggedright
\begin{center}
\scalebox{.560}{\showboard}
\end{center}
\newgame
%FEN@rnbqkb1r/1p3ppp/p3pn2/2p5/2BP4/4PN2/PP3PPP/RNBQ1RK1 w kq - 0 7
\openingname{Queen's Gambit Accepted}: \mainline{1. d4 d5 2. c4 dxc4}, main line: \mainline{3. Nf3 Nf6 4. e3 e6 5. Bxc4 c5 6. 0-0 a6} (diagram).\vspace{2mm}
\end{minipage}
\hspace{5mm}
\begin{minipage}[t]{.175\linewidth}
\fenboard{rnbqk2r/ppp1bppp/4pn2/3p2B1/2PP4/2N2N2/PP2PPPP/R2QKB1R b KQkq - 5 5}
\raggedright
\begin{center}
\scalebox{.560}{\showboard}
\end{center}
\newgame
%FEN@rnbqk2r/ppp1bppp/4pn2/3p2B1/2PP4/2N2N2/PP2PPPP/R2QKB1R b KQkq - 5 5
\openingname{Queen's Gambit Declined}: \mainline{1. d4 d5 2. c4 e6}, main line: \mainline{3. Nc3 Nf6 4. Bg5 Be7 5. Nf3} (diagram).\vspace{2mm}
\end{minipage}
\hspace{5mm}
\begin{minipage}[t]{.175\linewidth}
\fenboard{rnbqkb1r/pp2pppp/2p2n2/3p4/2PP4/5N2/PP2PPPP/RNBQKB1R w KQkq - 2 4}
\raggedright
\begin{center}
\scalebox{.560}{\showboard}
\end{center}
\newgame
%FEN@rnbqkb1r/pp2pppp/2p2n2/3p4/2PP4/5N2/PP2PPPP/RNBQKB1R w KQkq - 2 4
\openingname{Slav Defence}: \mainline{1. d4 d5 2. c4 c6}, main line: \mainline{3. Nf3 Nf6} (diagram) followed by \mainline{4. e3} or \variation{4. Nc3}.\vspace{2mm}
\end{minipage}
\hspace{5mm}
\begin{minipage}[t]{.175\linewidth}
\fenboard{r1bqkb1r/pp2pppp/2n2n2/2pp4/3P1P2/2PBP3/PP4PP/RNBQK1NR b KQkq f3 0 5}
\raggedright
\begin{center}
\scalebox{.560}{\showboard}
\end{center}
\newgame
%FEN@r1bqkb1r/pp2pppp/2n2n2/2pp4/3P1P2/2PBP3/PP4PP/RNBQK1NR b KQkq f3 0 5
\openingname{Stonewall Attack}: \mainline{1. d4 d5 2. e3 Nf6 3. Bd3 c5 4. c3 Nc6 5. f4} (regardless of how Black defends!).\vspace{2mm}
\end{minipage}
\hspace{5mm}
\begin{minipage}[t]{.175\linewidth}
\fenboard{r1bq1rk1/pp3ppp/2nbpn2/2pp4/3P4/2PBPN2/PP1N1PPP/R1BQR1K1 b - - 2 8}
\raggedright
\begin{center}
\scalebox{.560}{\showboard}
\end{center}
\newgame
%FEN@r1bq1rk1/pp3ppp/2nbpn2/2pp4/3P4/2PBPN2/PP1N1PPP/R1BQR1K1 b - - 2 8
\openingname{Colle System}: \mainline{1. d4 Nf6 2. Nf3 d5 3. e3 e6 4. Bd3 c5 5. O-O Nc6 6. Re1 Bd6 7. c3 O-O 8. Nbd2} (regardless of how Black defends!).\vspace{2mm}
\end{minipage}
\newline

\section{Semi-open games}
\begin{minipage}[t]{.175\linewidth}
\fenboard{rnbqkb1r/pp2pppp/3p1n2/8/3NP3/2N5/PPP2PPP/R1BQKB1R b KQkq - 2 5}
\raggedright
\begin{center}
\scalebox{.560}{\showboard}
\end{center}
\newgame
%FEN@rnbqkb1r/pp2pppp/3p1n2/8/3NP3/2N5/PPP2PPP/R1BQKB1R b KQkq - 2 5
\openingname{Sicilian Defence}: \mainline{1. e4 c5}, usually followed by \mainline{2. Nf3}.
After \mainline{2...d6 3. d4 cxd4 4. Nxd4 Nf6 5. Nc3} (diagram) Black can choose between four major variations:
\openingname{Najdorf} \variation{5...a6},
\openingname{Dragon} \variation{5...g6},
\openingname{Classical} \variation{5...Nc6} or
\openingname{Scheveningen} \variation{5...e6}.
\vspace{2mm}
\end{minipage}
\hspace{5mm}
\begin{minipage}[t]{.175\linewidth}
\fenboard{r1bqkbnr/pp1ppppp/2n5/8/3NP3/8/PPP2PPP/RNBQKB1R b KQkq - 0 4}
\raggedright
\begin{center}
\scalebox{.560}{\showboard}
\end{center}
\newgame
%FEN@r1bqkbnr/pp1ppppp/2n5/8/3NP3/8/PPP2PPP/RNBQKB1R b KQkq - 0 4
\openingname{Sicilian Defence}: \mainline{1. e4 c5}.
After \mainline{2. Nf3 Nc6 3. d4 cxd4 4. Nxd4} (diagram) common move is \mainline{4...Nf6}.
Other important moves are:
\openingname{transposed Taimanov Variation} \variation{4...e6},
\openingname{Accelerated Dragon} \variation{4...g6} and
\openingname{Kalashnikov Variation} \variation{4...e5}.
\vspace{2mm}
\end{minipage}
\hspace{5mm}
\begin{minipage}[t]{.175\linewidth}
\fenboard{rnbqkbnr/pp1p1ppp/4p3/8/3NP3/8/PPP2PPP/RNBQKB1R b KQkq - 0 4}
\raggedright
\begin{center}
\scalebox{.560}{\showboard}
\end{center}
\newgame
%FEN@rnbqkbnr/pp1p1ppp/4p3/8/3NP3/8/PPP2PPP/RNBQKB1R b KQkq - 0 4
\openingname{Sicilian Defence}: \mainline{1. e4 c5}.
After \mainline{2. Nf3 e6 3. d4 cxd4 4. Nxd4} (diagram) Black has three main moves:
\openingname{Taimanov Variation} \variation{4...Nc6},
\openingname{Kan Variation} \variation{4...a6} and 
\variation{4...Nf6}.
\vspace{2mm}
\end{minipage}
\hspace{5mm}
\begin{minipage}[t]{.175\linewidth}
\fenboard{r1bq1rk1/pp2ppbp/2np1np1/8/2BNP3/2N1BP2/PPPQ2PP/R3K2R b KQ - 4 9}
\raggedright
\begin{center}
\scalebox{.560}{\showboard}
\end{center}
\newgame
%FEN@r1bq1rk1/pp2ppbp/2np1np1/8/2BNP3/2N1BP2/PPPQ2PP/R3K2R b KQ - 4 9
\openingname{Sicilian Defence}: \mainline{1. e4 c5}.
\openingname{Dragon Variation}: \mainline{2. Nf3 d6 3. d4 cxd4 4. Nxd4 Nf6 5. Nc3 g6}.
\openingname{Yugoslav Attack}: \mainline{6. Be3 Bg7 7. f3 O-O 8. Qd2 Nc6 9. Bc4}.
Main line: \mainline{9... Bd7 10. 0-0-0 Rc8 11. Bb3 Ne5 12. Kb1 Re8} (diagram).

\vspace{2mm}
\end{minipage}
\hspace{5mm}
\begin{minipage}[t]{.175\linewidth}
\fenboard{rnbqk1nr/pp3ppp/4p3/2ppP3/3P4/P1P5/2P2PPP/R1BQKBNR b KQkq - 0 6}
\raggedright
\begin{center}
\scalebox{.560}{\showboard}
\end{center}
\newgame
%FEN@rnbqk1nr/pp3ppp/4p3/2ppP3/3P4/P1P5/2P2PPP/R1BQKBNR b KQkq - 0 6
\openingname{French Defence}: \mainline{1. e4 e6}, usually followed by \mainline{2. d4 d5 3. Nc3}.
Black has three options:
\openingname{Winawer} \mainline{3...Bb4}, main line then is: \mainline{4. e5 c5 5. a3 Bxc3+ 6. bxc3} (diagram);
\openingname{Classical} \variation{3...Nf6};
\openingname{Rubinstein} \variation{3...dxe4}.\vspace{2mm}
\end{minipage}
\newline
\begin{minipage}[t]{.175\linewidth}
\fenboard{rnbqk2r/ppp2ppp/4pb2/8/3PN3/5N2/PPP2PPP/R2QKB1R b KQkq - 1 7}
\raggedright
\begin{center}
\scalebox{.560}{\showboard}
\end{center}
\newgame
%FEN@rnbqk2r/ppp2ppp/4pb2/8/3PN3/5N2/PPP2PPP/R2QKB1R b KQkq - 1 7
\openingname{French Defence}: \mainline{1. e4 e6 2. d4 d5 3. Nc3}.
\openingname{Burn variation} is a line in classical variation: \mainline{3...Nf6 4. Bg5 dxe4 5. Nxe4}, usually followed by \mainline{5...Be7 6. Bxf6 Bxf6 7. Nf3} (diagram).\vspace{2mm}
\end{minipage}
\hspace{5mm}
\begin{minipage}[t]{.175\linewidth}
\fenboard{rnbqkbnr/pp2pppp/2p5/3p4/3PP3/2N5/PPP2PPP/R1BQKBNR b KQkq - 1 3}
\raggedright
\begin{center}
\scalebox{.560}{\showboard}
\end{center}
\newgame
%FEN@rnbqkbnr/pp2pppp/2p5/3p4/3PP3/2N5/PPP2PPP/R1BQKBNR b KQkq - 1 3
\openingname{Caro-Kann Defence}: \mainline{1. e4 c6}, after \mainline{2. d4 d5} common moves are \mainline{3. Nc3} (diagram), \variation{3. Nd2}, \variation{3. exd5} and \variation{3. e5}.\vspace{2mm}
\end{minipage}
\hspace{5mm}
\begin{minipage}[t]{.175\linewidth}
\fenboard{rnbq1rk1/ppp1ppbp/3p1np1/8/3PPP2/2N2N2/PPP3PP/R1BQKB1R w KQ - 3 6}
\raggedright
\begin{center}
\scalebox{.560}{\showboard}
\end{center}
\newgame
%FEN@rnbq1rk1/ppp1ppbp/3p1np1/8/3PPP2/2N2N2/PPP3PP/R1BQKB1R w KQ - 3 6
\openingname{Pirc Defence}: \mainline{1. e4 d6 2. d4 Nf6 3. Nc3 g6} followed by \mainline{4. f4 Bg7 5. Nf3 O-O} (\openingname{Austrian attack}, diagram) or \variation{4. Nf3 Bg7 5. Be2 O-O 6. O-O} (\openingname{two knights system}).\vspace{2mm}
\end{minipage}
\hspace{5mm}
\begin{minipage}[t]{.175\linewidth}
\fenboard{rnb1kbnr/ppp1pppp/8/q7/8/2N5/PPPP1PPP/R1BQKBNR w KQkq - 2 4}
\raggedright
\begin{center}
\scalebox{.560}{\showboard}
\end{center}
\newgame
%FEN@rnb1kbnr/ppp1pppp/8/q7/8/2N5/PPPP1PPP/R1BQKBNR w KQkq - 2 4
\openingname{Scandinavian Defence}: \mainline{1. e4 d5 2. exd5} followed by \mainline{2...Qxd5 3. Nc3} and then \mainline{3...Qa5} (diagram) or \variation{3...Qd8} or \variation{3...Qd6}. 
Another main branch is \variation{2...Nf6 3. d4 Nxd5 4. c4},\vspace{2mm}
\end{minipage}
\hspace{5mm}
\begin{minipage}[t]{.175\linewidth}
\fenboard{rn1qk1nr/pp2ppbp/2pp2p1/8/3PPPb1/2N2N2/PPP3PP/R1BQKB1R w KQkq - 2 6}
\raggedright
\begin{center}
\scalebox{.560}{\showboard}
\end{center}
\newgame
%FEN@rn1qk1nr/pp2ppbp/2pp2p1/8/3PPPb1/2N2N2/PPP3PP/R1BQKB1R w KQkq - 2 6
\openingname{Modern Defence}: \mainline{1. e4 g6} followed by \mainline{2. d4 Bg7 3. Nc3 d6 4. f4 c6 5. Nf3 Bg4}.

\vspace{2mm}
\end{minipage}
\newline

\section{Indian defences}
\begin{minipage}[t]{.175\linewidth}
\fenboard{rnbqkb1r/pp3p1p/3p1np1/2pP4/8/2N2N2/PP2PPPP/R1BQKB1R w KQkq - 0 7}
\raggedright
\begin{center}
\scalebox{.560}{\showboard}
\end{center}
\newgame
%FEN@rnbqkb1r/pp3p1p/3p1np1/2pP4/8/2N2N2/PP2PPPP/R1BQKB1R w KQkq - 0 7
\openingname{Modern Benoni}: \mainline{1. d4 Nf6 2. c4 c5 3. d5 e6}, can be followed by \mainline{4. Nc3 exd5 5. cxd5 d6 6. Nf3 g6} and then \openingname{classical}: \mainline{7. Nf3 Bg7 8. Be2 O-O 9. O-O} (diagram) or \openingname{modern} main line: \variation{7. Nf3 Bg7 8. h3 O-O 9. Bd3}.\vspace{2mm}
\end{minipage}
\hspace{5mm}
\begin{minipage}[t]{.175\linewidth}
\fenboard{rnbqkb1r/pp3p1p/3p1np1/2pP4/4P3/2N5/PP3PPP/R1BQKBNR w KQkq - 0 7}
\raggedright
\begin{center}
\scalebox{.560}{\showboard}
\end{center}
\newgame
%FEN@rnbqkb1r/pp3p1p/3p1np1/2pP4/4P3/2N5/PP3PPP/R1BQKBNR w KQkq - 0 7
\openingname{Modern Benoni}: \mainline{1. d4 Nf6 2. c4 c5 3. d5 e6}, can be followed by \mainline{4. Nc3 exd5 5. cxd5 d6 6. e4 g6} (diagram).\vspace{2mm}
\end{minipage}
\hspace{5mm}
\begin{minipage}[t]{.175\linewidth}
\fenboard{rn1qkb1r/3ppppp/b4n2/2pP4/8/8/PP2PPPP/RNBQKBNR w KQkq - 0 6}
\raggedright
\begin{center}
\scalebox{.560}{\showboard}
\end{center}
\newgame
%FEN@rn1qkb1r/3ppppp/b4n2/2pP4/8/8/PP2PPPP/RNBQKBNR w KQkq - 0 6
\openingname{Benko/Volga Gambit}: \mainline{1. d4 Nf6 2. c4 c5 3. d5 b5}, main line: \mainline{4. cxb5 a6 5. bxa6 Bxa6} (diagram).\vspace{2mm}
\end{minipage}
\hspace{5mm}
\begin{minipage}[t]{.175\linewidth}
\fenboard{rnbq1rk1/pp3ppp/4pn2/2pp4/1bPP4/2NBPN2/PP3PPP/R1BQ1RK1 b - - 1 7}
\raggedright
\begin{center}
\scalebox{.560}{\showboard}
\end{center}
\newgame
%FEN@rnbq1rk1/pp3ppp/4pn2/2pp4/1bPP4/2NBPN2/PP3PPP/R1BQ1RK1 b - - 1 7
\openingname{Nimzo-Indian Defence}: \mainline{1. d4 Nf6 2. c4 e6 3. Nc3 Bb4}, common reply is \openingname{Rubinstein System}: \mainline{4. e3} with main line: \mainline{4...O-O 5. Bd3 d5 6. Nf3 c5 7. O-O} (diagram).\vspace{2mm}
\end{minipage}
\hspace{5mm}
\begin{minipage}[t]{.175\linewidth}
\fenboard{rn1qkb1r/p1pp1ppp/bp2pn2/8/2PP4/5NP1/PP2PP1P/RNBQKB1R w KQkq - 1 5}
\raggedright
\begin{center}
\scalebox{.560}{\showboard}
\end{center}
\newgame
%FEN@rn1qkb1r/p1pp1ppp/bp2pn2/8/2PP4/5NP1/PP2PP1P/RNBQKB1R w KQkq - 1 5
\openingname{Queen's Indian Defence}: \mainline{1. d4 Nf6 2. c4 e6 3. Nf3 b6}, main line: \mainline{4. g3 Ba6} (diagram).
\vspace{2mm}
\end{minipage}
\newline
\begin{minipage}[t]{.175\linewidth}
\fenboard{rnbqk2r/ppp1bppp/4pn2/8/2pP4/5NP1/PP2PPBP/RNBQK2R w KQkq - 2 6}
\raggedright
\begin{center}
\scalebox{.560}{\showboard}
\end{center}
\newgame
%FEN@rnbqk2r/ppp1bppp/4pn2/8/2pP4/5NP1/PP2PPBP/RNBQK2R w KQkq - 2 6
\openingname{Catalan Opening}: \mainline{1. d4 Nf6 2. c4 e6 3. g3} with \openingname{classical} line: \mainline{3...d5 4. Bg2 dxc4 5. Nf3 Be7} (diagram).\vspace{2mm}
\end{minipage}
\hspace{5mm}
\begin{minipage}[t]{.175\linewidth}
\fenboard{rnbqkb1r/ppp1pp1p/6p1/8/3PP3/2P5/P4PPP/R1BQKBNR b KQkq - 0 6}
\raggedright
\begin{center}
\scalebox{.560}{\showboard}
\end{center}
\newgame
%FEN@rnbqkb1r/ppp1pp1p/6p1/8/3PP3/2P5/P4PPP/R1BQKBNR b KQkq - 0 6
\openingname{Grünfeld Defence}: \mainline{1. d4 Nf6 2. c4 g6 3. Nc3 d5}.
\openingname{Exchange Variation}: \mainline{4. cxd5 Nxd5 5. e4 Nxc3 6. bxc3} (diagram).
\vspace{2mm}
\end{minipage}
\hspace{5mm}
\begin{minipage}[t]{.175\linewidth}
\fenboard{rnbq1rk1/ppp1ppbp/5np1/8/2QPP3/2N2N2/PP3PPP/R1B1KB1R b KQ e3 0 7}
\raggedright
\begin{center}
\scalebox{.560}{\showboard}
\end{center}
\newgame
%FEN@rnbq1rk1/ppp1ppbp/5np1/8/2QPP3/2N2N2/PP3PPP/R1B1KB1R b KQ e3 0 7
\openingname{Grünfeld Defence}: \mainline{1. d4 Nf6 2. c4 g6 3. Nc3 d5}.
\openingname{Russian system}: \mainline{4. Nf3 Bg7 5. Qb3 dxc4 6. Qxc4 O-O 7. e4} (diagram).
Another is \openingname{Taimanov's Variation} \variation{4. Nf3 Bg7 5. Bg5}.\vspace{2mm}
\end{minipage}
\hspace{5mm}
\begin{minipage}[t]{.175\linewidth}
\fenboard{rnbq1rk1/ppp2pbp/3p1np1/4p3/2PPP3/2N2N2/PP2BPPP/R1BQK2R w KQ e6 0 7}
\raggedright
\begin{center}
\scalebox{.560}{\showboard}
\end{center}
\newgame
%FEN@rnbq1rk1/ppp2pbp/3p1np1/4p3/2PPP3/2N2N2/PP2BPPP/R1BQK2R w KQ e6 0 7
\openingname{King's Indian Defence}: \mainline{1. d4 Nf6 2. c4 g6}, \openingname{Classical Variation}: \mainline{3. Nc3 Bg7 4. e4 d6 5. Nf3 O-O 6. Be2 e5} (diagram).\vspace{2mm}
\end{minipage}
\hspace{5mm}
\begin{minipage}[t]{.175\linewidth}
\fenboard{rnbq1rk1/ppp1ppbp/3p1np1/8/2PP4/5NP1/PP2PPBP/RNBQ1RK1 b - - 1 6}
\raggedright
\begin{center}
\scalebox{.560}{\showboard}
\end{center}
\newgame
%FEN@rnbq1rk1/ppp1ppbp/3p1np1/8/2PP4/5NP1/PP2PPBP/RNBQ1RK1 b - - 1 6
\openingname{King's Indian Defence}: \mainline{1. d4 Nf6 2. c4 g6}, \openingname{Fianchetto Variation}: \mainline{3. Nf3 Bg7 4. g3 O-O 5. Bg2 d6 6. O-O} (diagram).\vspace{2mm}
\end{minipage}
\newline

\section{Flank openings}
\begin{minipage}[t]{.175\linewidth}
\fenboard{rnbqkbnr/pppppppp/8/8/2P5/8/PP1PPPPP/RNBQKBNR b KQkq c3 0 1}
\raggedright
\begin{center}
\scalebox{.560}{\showboard}
\end{center}
\newgame
%FEN@rnbqkbnr/pppppppp/8/8/2P5/8/PP1PPPPP/RNBQKBNR b KQkq c3 0 1
\openingname{English Opening}: \mainline{1. c4} (diagram).
Then \mainline{1...Nf6} \openingname{Anglo-Indian Defence}, \variation{1...e5} \openingname{Reversed Sicilian}, \variation{1...e6} \openingname{Agincourt Defence} or \variation{1...c5} \openingname{Symmetrical Variation}.

\vspace{2mm}
\end{minipage}
\hspace{5mm}
\begin{minipage}[t]{.175\linewidth}
\fenboard{rnbqk1nr/ppp2ppp/3b4/8/8/8/PPPPP1PP/RNBQKBNR w KQkq - 0 4}
\raggedright
\begin{center}
\scalebox{.560}{\showboard}
\end{center}
\newgame
%FEN@rnbqk1nr/ppp2ppp/3b4/8/8/8/PPPPP1PP/RNBQKBNR w KQkq - 0 4
\openingname{Bird's Opening}: \mainline{1. f4}.

\openingname{Leningrad Variation}: \mainline{1...d5 2. Nf3 g6 3. g3 Bg7 4. Bg2}.
\openingname{From's Gambit}: \variation{1...e5 2. fxe5 d6 3. exd6 Bxd6} (diagram).
\vspace{2mm}
\end{minipage}
\hspace{5mm}


\end{document}
